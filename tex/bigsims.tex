%% LyX 2.0.3 created this file.  For more info, see http://www.lyx.org/.
%% Do not edit unless you really know what you are doing.
\documentclass[usenatbib]{mn2e}
\usepackage[latin9]{inputenc}
\usepackage[a4paper]{geometry}
\geometry{verbose}
\setcounter{tocdepth}{3}
\usepackage{graphicx}

\makeatletter

%%%%%%%%%%%%%%%%%%%%%%%%%%%%%% LyX specific LaTeX commands.
%% A simple dot to overcome graphicx limitations
\newcommand{\lyxdot}{.}


%%%%%%%%%%%%%%%%%%%%%%%%%%%%%% User specified LaTeX commands.






%%%%%%%%%%%%%%%%%%%%%%%%%%%%%% LyX specific LaTeX commands.
%% A simple dot to overcome graphicx limitations
%Make my life significantly easier

\global\long\def\bd{{\bm{\delta}}}

\makeatother

\begin{document}
\title[Mock catalogs for BOSS]{Generating Mock Catalogs for the Baryon Oscillation Spectroscopic Survey : An Approximate N-Body approach}\author[People??]{Aaronson$^{1}$,Aardvark$^{1}$\\
$^{1}$Universe
}

\maketitle 
\begin{abstract}
We introduce and test an approximate scheme for generating mock catalogs
for large-scale structure measurements in galaxy surveys, specializing
in this work to the Baryon Oscillation Spectroscopic Survey. A brief
description of the approximation scheme goes here??? A brief description
of the tests and the accuracy we reach goes here??? Some comments
about the timings of the tests goes here??? Final comments about the
BOSS samples go here???? 
\end{abstract}

\section{Introduction}

Why do we need large numbers of large N-body simulations? Discuss
both covariance matrix estimation as well as systematic error estimation.
Emphasize importance to capture as much of the physics as is possible.

What about just running large numbers of N-body simulations : surveys
are getting larger, and the numbers of simulations required is also
large for covariance matrices. Ideally also require variations in
cosmological parameters. Mention the LasDamas suite of simulations
here as exemplars of this approach.

Other approaches discussed here : Gaussian and log-normal random simulations;
issues with these.

2LPT, PTHalos, Pinocchio : discuss recent work by Manera et al. Also
reference work in progress by Tassev et al.

Our approach -- tune down N-body simulations.

A brief discussion of BOSS here and the need for large N-body simulations
there.

An outline of the paper needs to go here\ldots{}


\section{Generating Approximate Simulations}

A brief discussion of the HACC code goes here, pointing out all the
great things it can do.

Focus in on the key aspects of the code that we use for the approximate
scheme : time steps on the PM side, and subcycles. Explain the parameters
we tune.


\section{Convergence and Timing Tests}

Describe the suites of simulations we run for convergence tests.

(a) 256 Mpc boxes. How many do we need? Which approximation parameters
should we choose to vary? What are our references -- full N-body with
same particle loading, full N-body with higher particle loading? Emulator?

(b) do we need any other box sizes? What do we hope to learn from
these?


\subsection{Dark Matter}

What tests do we need here? A minimum list I can think of 
\begin{itemize}
\item Convergence of the power spectrum as a function of approximation and
redshift. 
\item Cross-correlation between full and approximate N-body. 
\end{itemize}

\subsection{Halos}

Questions that we want to answer here are

1) how well do different simulations agree with each other?

2) how do those different simulations affect on observables (i.e.,
mass function, power spectra,...e.t.c)?

3) What mass resolution and timesteps are reasonable to generate mock
catalogs? (This requires a reasonable agreement with results from
full N-body simulations.)

-> I would like to compare halo samples from full N-body simulations
and the samples from 450/5 with $512^{3}$ particles to see how much
deviation there are.
\begin{itemize}
\item Mass function comparisons: 
\end{itemize}
0) What is mass functions? Why do we care?

1) How does mass resolution affect on mass function at different redshifts?

2) How does different stepsizes affect on mass function?

\begin{figure}
\includegraphics[width=0.5\columnwidth]{/Users/ts485/Dropbox/Tomomi/HACC/Conv/Plots4lyx/haloNum2}

\caption{Comparison of mass functions (should I call this number density?)
for different simulations at redshift $z=0.15$ and $z=0.8$. The
shaded regions indicate the upper limit and the lower limit of mass
functions for the simulation with $512^{3}$ particles and 450 steps
with 5 inner(?) steps. The mean and standard deviations are calculated
from bootstrap method. We compare them with the simulations of $256^{3}$
particles with 450 large steps and 5 inner steps (described as $256^{3}(450:5)$
with circles in the figure) and with 300 large steps and 2 inner steps
(described as $256^{3}(300:2)$ with red stars in the figure).}
\end{figure}
 
\begin{figure}
\includegraphics[width=0.5\columnwidth]{/Users/ts485/Dropbox/Tomomi/HACC/Conv/Plots4lyx/haloRatioNum256_z0\lyxdot 15}\caption{Ratios of mass functions of different time steps with $256^{3}$ particles,
compared to $512^{3}$ particles with 450 large steps and 5 inner
steps, at redshift $z=0.15$. Different colors are corresponding to
different time steps. The legend in the plot has the same format as
the one in Figure 1. This plot shows that the simulations with 450
and 300 large steps agree well with the simulation with $512^{3}$
particles.}
\end{figure}

\begin{itemize}
\item Halo power spectra and bias: 
\begin{figure}
\includegraphics[width=0.5\columnwidth]{/Users/ts485/Dropbox/Tomomi/HACC/Conv/Plots4lyx/autoPower_mcut12\lyxdot 5_z0\lyxdot 15}\includegraphics[width=0.5\columnwidth]{/Users/ts485/Dropbox/Tomomi/HACC/Conv/Plots4lyx/autoPower_ratio_mcut12\lyxdot 5_z0\lyxdot 15}

\caption{Left: Halo auto-power spectra obtained from different particle numbers
and simulation step sizes. For each of the simulations: $512^{3}$
particles with 450 large steps and 5 inner steps (line), $256^{3}$
particles with 450 large steps and 5 inner steps (circle), and $256^{3}$
particles with 300 large steps and 2 inner steps (cross). Right: Ratios
of halo auto-power spectra of $256^{3}$ particles and different simulation
step sizes with respect to the power spectrum of $512^{3}$ particles
with 450 large steps and 5 inner steps. Different colors corresponds
to different stepsizes: 450 large steps and 5 inner steps (blue),
300 large steps and 3 inner steps (green), 300 large steps and 2 inner
steps (red), 150 large steps and 3 inner steps (cyan), and 150 large
steps and 2 inner steps (purple). }
\end{figure}
 
\begin{figure}
\includegraphics[width=0.5\columnwidth]{/Users/ts485/Dropbox/Tomomi/HACC/Conv/Plots4lyx/Bias_mass_ws_mcut12\lyxdot 5to13\lyxdot 0_z0\lyxdot 15}\includegraphics[width=0.5\columnwidth]{/Users/ts485/Dropbox/Tomomi/HACC/Conv/Plots4lyx/Bias_matched_ws_mcut12\lyxdot 5to13\lyxdot 0_z0\lyxdot 15}

\caption{Halo Biases for different stepsizes with $256^{3}$ particles at redshift
$z=0.15$. Left panel is halo biases for samples which are selected
based on mass, and right panel is for samples which are corresponding
to halos of 450 large steps and 5 inner steps. Different colors indicate
different stepsizes and mass range for halo samples is from $10^{12.5}{\rm M_{\odot}}$
to $10^{13.0}{\rm M_{\odot}}$.}
\end{figure}

\item Cross correlations with full N-body, higher particle loading etc 
\begin{figure}
\includegraphics[width=0.5\columnwidth]{/Users/ts485/Dropbox/Tomomi/HACC/Conv/Plots4lyx/crossMatter4_z0\lyxdot 15}\includegraphics[width=0.5\columnwidth]{/Users/ts485/Dropbox/Tomomi/HACC/Conv/Plots4lyx/crossMatter3_ratio_mcut12\lyxdot 5to13\lyxdot 0_z0\lyxdot 15}

\caption{Left: Cross power spectra of halos with DM particles at redshift $z=0.15$.
DM particles are taken from the simulation of $256^{3}$ particles
with 450 large steps and 5 inner steps. Different colors indicate
different halo mass slices: ${\rm log_{10}M\in[12.5,13.0]}$ (blue),
${\rm log_{10}M\in[13.0,13.5]}$ (green), and ${\rm log_{10}M>13.5}$
(red). Lines are the simulations with 450 large steps and 5 inner
steps, and circles are the ones with 300 large steps and 2 inner steps.
Right: Ratios of cross power spectra for different simulations with
respect to the cross power spectra with 450 large steps and 5 inner
steps. Both cross power spectra are with respect to DM particles,
which is the same as the left plot.}
\end{figure}

\item Halo position, mass and velocity comparisons (of what?): Halos for
different simulations are selected so that they correspond to halos
in the simulation of $256^{3}$ particles with 450 large steps and
5 inner steps. The selection criteria are 1) two halos are within
0.5 $h^{-1}{\rm Mpc}$, and 2) a log-based mass difference of two
halos are within -0.5 to 0.5. 
\begin{figure}
\includegraphics[width=0.5\columnwidth]{/Users/ts485/Dropbox/Tomomi/HACC/Conv/Plots4lyx/histogram3_diff_z0\lyxdot 15}

\caption{Histograms of log-based mass difference of different simulation stepsizes
with respect to the one with 450 large steps and 5 inner steps. Histograms
are not normalized and all halos are from the simulations with $256^{3}$
particles. For each of simulations: 300 large steps with 3 inner steps
(blue) and 2 inner steps (green), and 150 large steps with 3 inner
steps (red).}
\end{figure}
\begin{figure}
\includegraphics[width=0.5\columnwidth]{/Users/ts485/Dropbox/Tomomi/HACC/Conv/Plots4lyx/distance1_mcut12\lyxdot 5to13\lyxdot 0_z0\lyxdot 15}

\caption{Histograms of distance difference for matched halos with respect to
halos from $256^{3}$ particles with 450 large steps and 5 inner steps.
Different colors correspond to different simulation stepsizes: 300
large steps with 3 inner steps (blue) and 2 inner steps (green), and
150 large steps with 3 inner steps (red) and 2 inner steps (cyan).
All the simulations use $256^{3}$ particles at redshift $z=0.15$.}
\end{figure}
\begin{figure}
\includegraphics[width=0.5\columnwidth]{/Users/ts485/Dropbox/Tomomi/HACC/Conv/Plots4lyx/velMag1_z0\lyxdot 15}

\caption{Histograms of velocity magnitude difference for matched halos with
respect to halos from $256^{3}$ particles with 450 large steps and
5 inner steps. Different colors correspond to different simulation
stepsizes: 300 large steps with 3 inner steps (blue) and 2 inner steps
(green), and 150 large steps with 3 inner steps (red) and 2 inner
steps (cyan). All the simulations use $256^{3}$ particles at redshift
$z=0.15$. Note that angle difference of velocity vectors for 98\%
of matched halos are within 0.3 radians.}
\end{figure}
\begin{figure}
\includegraphics[width=0.5\columnwidth]{/Users/ts485/Dropbox/Tomomi/HACC/Conv/Plots4lyx/scatter2_256_300_2_z0\lyxdot 15}

\caption{Comparison of halo masses for the samples (which is $256^{3}$particles
simulation with 300 large steps and 2 inner steps) which are selected
so that they match with the halos in the smulation of 450 large steps
and 5 inner steps. The x-axis is log-based halo masses for the 450-step
simulation and the y-axis is for the 300-step simulation. Both samples
are at $z=0.15$.}
\end{figure}
\begin{figure}
\includegraphics[width=0.5\columnwidth]{/Users/ts485/Dropbox/Tomomi/HACC/Conv/Plots4lyx/New_unmatchedHalo_256_300_2_z0\lyxdot 15}\includegraphics[width=0.5\columnwidth]{/Users/ts485/Dropbox/Tomomi/HACC/Conv/Plots4lyx/New_unmatchedHaloRatio_256_300_2_z0\lyxdot 15}

\caption{Left: Unmatched halo number density for the simulation of 300 large
steps and 2 inner steps matching with 450 large steps and 5 inner
steps. Both are from $256^{3}$ particle simulations. Right: Ratio
of unmatched halo number densities (which are the same as the ones
in the left plot) with respect to the corresponding total number densities.
Both plots are at redshift $z=0.15$.}
\end{figure}

\item Comparison of matched halo samples and mass-sliced samples: what the
comparison indicates is that corresponding halos don't have the same
masses. ->What kind of problems do we have by having this issue?->$b(M)$
may be diffrent for different simulations. 
\begin{figure}
\includegraphics[width=0.5\columnwidth]{/Users/ts485/Dropbox/Tomomi/HACC/Conv/Plots4lyx/autoPower_ratio256_mcut12\lyxdot 5to13\lyxdot 0_z0\lyxdot 15}\includegraphics[width=0.5\columnwidth]{/Users/ts485/Dropbox/Tomomi/HACC/Conv/Plots4lyx/matchPower2_wo_mcut12\lyxdot 5to13\lyxdot 0_z0\lyxdot 15}

\caption{Those are the ratios of auto-power spectra of different stepsizes
in $256^{3}$ particle sinulations. Halos in the left plot are selected
based on mass, while halos in the right plot are chosen so that they
match with halos with 450 large steps and 5 inner steps. I need to
evaluate the effect of shot noises for each plot to see why the deviations
of 150-step simulations are larger for matched halos...}
\end{figure}
\begin{figure}
\includegraphics[width=0.5\columnwidth]{/Users/ts485/Dropbox/Tomomi/HACC/Conv/Plots4lyx/crossCoeff2_mcut12\lyxdot 5to13\lyxdot 0_z0\lyxdot 15}\includegraphics[width=0.5\columnwidth]{/Users/ts485/Dropbox/Tomomi/HACC/Conv/Plots4lyx/matchedCross1_mcut12\lyxdot 5to13\lyxdot 0_z0\lyxdot 15}

\caption{Those are the cross-correlation coefficients of different stepsizes
in $256^{3}$ particle sinulations. Halos in the left plot are selected
based on mass, while halos in the right plot are chosen so that they
match with halos with 450 large steps and 5 inner steps. I feel that
this plot somehow contradicts with Figure 11 (though this result makes
sense).}
\end{figure}

\end{itemize}

\section{The BOSS Simulations}


\subsection{Simulation Parameters}

describe box size, masses, geometry etc. Show that we can fit in two
BOSS volumes per box.


\subsection{Building the Galaxy Catalogs}

How to do the redshift evolution -- interpolating between different
redshift snapshots

HODs at least for the basic BOSS sims.


\subsection{An Application}

Maybe something simple here.


\section{Discussion}

Some words on conclusions go here. 
\end{document}
