%% LyX 2.0.3 created this file.  For more info, see http://www.lyx.org/.
%% Do not edit unless you really know what you are doing.
\documentclass[usenatbib]{mn2e}
\usepackage[latin9]{inputenc}
\usepackage[a4paper]{geometry}
\geometry{verbose}
\setcounter{tocdepth}{3}
\usepackage{color}
\usepackage{graphicx}

\makeatletter

%%%%%%%%%%%%%%%%%%%%%%%%%%%%%% LyX specific LaTeX commands.
%% A simple dot to overcome graphicx limitations
\newcommand{\lyxdot}{.}


%%%%%%%%%%%%%%%%%%%%%%%%%%%%%% User specified LaTeX commands.






%%%%%%%%%%%%%%%%%%%%%%%%%%%%%% LyX specific LaTeX commands.
%% A simple dot to overcome graphicx limitations
%Make my life significantly easier

\global\long\def\bd{{\bm{\delta}}}

\makeatother

\begin{document}
We are interested in recovering the halos and their masses, positions
and velocities with the smallest time step necessary to preserve sufficient
properties as mock catalogs.


\section{Matching Halos}

First, we compare a halo by halo in different simulations. Since all
the simulations use exactly the same initial conditions, if those
approximated N-body simulations can recover the halos reasonably well
(compared to a full N-body simulation), then we should find the same
halo with the same mass at the same position in different samples.
The question we want to answer here are how mass resolutions and time
steps affect on halo properties (i.e., mass, position and velocity)
and how many halos don't match.

To spatially match the halos obtained in different mass resolution
and time step parameters, we first find a pair of halos from two different
simulations, whose distance is the closest. Then, we set two conditions
on those pairs to declare that those paired halos are the same halos:

1) distance is smaller than $0.5h^{-1}{\rm Mpc}$,

2) mass ratio is smaller than $10^{0.5}{\rm M_{\odot}}$.

Under the above conditions, \textcolor{red}{XX}\% of halos in 450/5
found paired halos in other samples (shown in \textcolor{red}{Table
Y}: fraction of matched halos as a function of distance and mass ratios+lower
halo mass limit). We also checked how those conditions affect to fraction
of matched halos by changing the numerical values for distance and
mass ratio criteria. As shown in \textcolor{red}{Table Y}, changing
a criterion on mass ratio did not change the fraction compared to
the distance criterion. This indicates that deviation of halo masses
on pairs is relatively small.


\subsection{Mass Resolution}

{*} We take a simulation of $512^{3}$ particles as a reference.


\subsection{Time steps}

{*} We vary time step parameters with a fixed mass resolution, $256^{3}$particles
in $(256h{\rm ^{-1}Mpc})^{3}$-volume box.

{*} We take 450/5 as a reference and compare other time steps (300
and 150 global time steps and 3 and 2 sub-cycles).
\begin{itemize}
\item Table: unmatched halos with different conditions
\item Histogram: position (center, mean, mix?) \& table for mean and standard
deviation 
\begin{figure}
\includegraphics[width=0.5\columnwidth]{/Users/ts485/Dropbox/Tomomi/HACC/Conv/Plots4lyx/distance1_mcut12\lyxdot 5to13\lyxdot 0_z0\lyxdot 15}

\caption{Histograms of distance difference for matched halos with respect to
halos from $256^{3}$ particles with 450 large steps and 5 inner steps.
Different colors correspond to different simulation stepsizes: 300
large steps with 3 inner steps (blue) and 2 inner steps (green), and
150 large steps with 3 inner steps (red) and 2 inner steps (cyan).
All the simulations use $256^{3}$ particles at redshift $z=0.15$.}
\end{figure}
(mean)
\item Histogram: Mass 
\begin{figure}
\includegraphics[width=0.5\columnwidth]{/Users/ts485/Dropbox/Tomomi/HACC/Conv/Plots4lyx/histogram3_diff_z0\lyxdot 15}

\caption{Histograms of log-based mass difference of different simulation stepsizes
with respect to the one with 450 large steps and 5 inner steps. Histograms
are not normalized and all halos are from the simulations with $256^{3}$
particles. For each of simulations: 300 large steps with 3 inner steps
(blue) and 2 inner steps (green), and 150 large steps with 3 inner
steps (red).}
\end{figure}
 
\begin{figure}
\includegraphics[width=0.5\columnwidth]{/Users/ts485/Dropbox/Tomomi/HACC/Conv/Plots4lyx/scatter2_256_300_2_z0\lyxdot 15}

\caption{Comparison of halo masses for the samples (which is $256^{3}$particles
simulation with 300 large steps and 2 inner steps) which are selected
so that they match with the halos in the smulation of 450 large steps
and 5 inner steps. The x-axis is log-based halo masses for the 450-step
simulation and the y-axis is for the 300-step simulation. Both samples
are at $z=0.15$.}
\end{figure}

\item Histogram: Velocity 
\begin{figure}
\includegraphics[width=0.5\columnwidth]{/Users/ts485/Dropbox/Tomomi/HACC/Conv/Plots4lyx/velMag1_z0\lyxdot 15}

\caption{Histograms of velocity magnitude difference for matched halos with
respect to halos from $256^{3}$ particles with 450 large steps and
5 inner steps. Different colors correspond to different simulation
stepsizes: 300 large steps with 3 inner steps (blue) and 2 inner steps
(green), and 150 large steps with 3 inner steps (red) and 2 inner
steps (cyan). All the simulations use $256^{3}$ particles at redshift
$z=0.15$. Note that angle difference of velocity vectors for 98\%
of matched halos are within 0.3 radians.}
\end{figure}

\item $n(M)$ for matched/unmatched halos (should I also put total?) 
\begin{figure}
\includegraphics[width=0.5\columnwidth]{/Users/ts485/Dropbox/Tomomi/HACC/Conv/Plots4lyx/New_unmatchedHalo_256_300_2_z0\lyxdot 15}\includegraphics[width=0.5\columnwidth]{/Users/ts485/Dropbox/Tomomi/HACC/Conv/Plots4lyx/New_unmatchedHaloRatio_256_300_2_z0\lyxdot 15}

\caption{Left: Unmatched halo number density for the simulation of 300 large
steps and 2 inner steps matching with 450 large steps and 5 inner
steps. Both are from $256^{3}$ particle simulations. Right: Ratio
of unmatched halo number densities (which are the same as the ones
in the left plot) with respect to the corresponding total number densities.
Both plots are at redshift $z=0.15$.}
\end{figure}

\item snapshots for unmatched halos whose mass is greater than $10^{14}{\rm M_{\odot}}$
\end{itemize}

\section{Observable/Statistics}
\begin{itemize}
\item Mass function comparisons: 
\begin{figure}
\includegraphics[width=0.5\columnwidth]{/Users/ts485/Dropbox/Tomomi/HACC/Conv/Plots4lyx/haloNum2}

\caption{Comparison of mass functions (should I call this number density?)
for different simulations at redshift $z=0.15$ and $z=0.8$. The
shaded regions indicate the upper limit and the lower limit of mass
functions for the simulation with $512^{3}$ particles and 450 steps
with 5 inner(?) steps. The mean and standard deviations are calculated
from bootstrap method. We compare them with the simulations of $256^{3}$
particles with 450 large steps and 5 inner steps (described as $256^{3}(450:5)$
with circles in the figure) and with 300 large steps and 2 inner steps
(described as $256^{3}(300:2)$ with red stars in the figure).}
\end{figure}
 
\begin{figure}
\includegraphics[width=0.5\columnwidth]{/Users/ts485/Dropbox/Tomomi/HACC/Conv/Plots4lyx/haloRatioNum256_z0\lyxdot 15}\caption{Ratios of mass functions of different time steps with $256^{3}$ particles,
compared to $512^{3}$ particles with 450 large steps and 5 inner
steps, at redshift $z=0.15$. Different colors are corresponding to
different time steps. The legend in the plot has the same format as
the one in Figure 1. This plot shows that the simulations with 450
and 300 large steps agree well with the simulation with $512^{3}$
particles.}
\end{figure}

\item Halo power spectra and bias: 
\begin{figure}
\includegraphics[width=0.5\columnwidth]{/Users/ts485/Dropbox/Tomomi/HACC/Conv/Plots4lyx/autoPower_mcut12\lyxdot 5_z0\lyxdot 15}\includegraphics[width=0.5\columnwidth]{/Users/ts485/Dropbox/Tomomi/HACC/Conv/Plots4lyx/autoPower_ratio_mcut12\lyxdot 5_z0\lyxdot 15}

\caption{Left: Halo auto-power spectra obtained from different particle numbers
and simulation step sizes. For each of the simulations: $512^{3}$
particles with 450 large steps and 5 inner steps (line), $256^{3}$
particles with 450 large steps and 5 inner steps (circle), and $256^{3}$
particles with 300 large steps and 2 inner steps (cross). Right: Ratios
of halo auto-power spectra of $256^{3}$ particles and different simulation
step sizes with respect to the power spectrum of $512^{3}$ particles
with 450 large steps and 5 inner steps. Different colors corresponds
to different stepsizes: 450 large steps and 5 inner steps (blue),
300 large steps and 3 inner steps (green), 300 large steps and 2 inner
steps (red), 150 large steps and 3 inner steps (cyan), and 150 large
steps and 2 inner steps (purple). }
\end{figure}
 
\begin{figure}
\includegraphics[width=0.5\columnwidth]{/Users/ts485/Dropbox/Tomomi/HACC/Conv/Plots4lyx/Bias_mass_ws_mcut12\lyxdot 5to13\lyxdot 0_z0\lyxdot 15}\includegraphics[width=0.5\columnwidth]{/Users/ts485/Dropbox/Tomomi/HACC/Conv/Plots4lyx/Bias_matched_ws_mcut12\lyxdot 5to13\lyxdot 0_z0\lyxdot 15}

\caption{Halo Biases for different stepsizes with $256^{3}$ particles at redshift
$z=0.15$. Left panel is halo biases for samples which are selected
based on mass, and right panel is for samples which are corresponding
to halos of 450 large steps and 5 inner steps. Different colors indicate
different stepsizes and mass range for halo samples is from $10^{12.5}{\rm M_{\odot}}$
to $10^{13.0}{\rm M_{\odot}}$.}
\end{figure}

\item Cross correlations with full N-body, higher particle loading etc 
\begin{figure}
\includegraphics[width=0.5\columnwidth]{/Users/ts485/Dropbox/Tomomi/HACC/Conv/Plots4lyx/crossMatter4_z0\lyxdot 15}\includegraphics[width=0.5\columnwidth]{/Users/ts485/Dropbox/Tomomi/HACC/Conv/Plots4lyx/crossMatter3_ratio_mcut12\lyxdot 5to13\lyxdot 0_z0\lyxdot 15}

\caption{Left: Cross power spectra of halos with DM particles at redshift $z=0.15$.
DM particles are taken from the simulation of $256^{3}$ particles
with 450 large steps and 5 inner steps. Different colors indicate
different halo mass slices: ${\rm log_{10}M\in[12.5,13.0]}$ (blue),
${\rm log_{10}M\in[13.0,13.5]}$ (green), and ${\rm log_{10}M>13.5}$
(red). Lines are the simulations with 450 large steps and 5 inner
steps, and circles are the ones with 300 large steps and 2 inner steps.
Right: Ratios of cross power spectra for different simulations with
respect to the cross power spectra with 450 large steps and 5 inner
steps. Both cross power spectra are with respect to DM particles,
which is the same as the left plot.}
\end{figure}

\item Comparison of matched halo samples and mass-sliced samples: what the
comparison indicates is that corresponding halos don't have the same
masses. ->What kind of problems do we have by having this issue?->$b(M)$
may be diffrent for different simulations. 
\begin{figure}
\includegraphics[width=0.5\columnwidth]{/Users/ts485/Dropbox/Tomomi/HACC/Conv/Plots4lyx/autoPower_ratio256_mcut12\lyxdot 5to13\lyxdot 0_z0\lyxdot 15}\includegraphics[width=0.5\columnwidth]{/Users/ts485/Dropbox/Tomomi/HACC/Conv/Plots4lyx/matchPower2_wo_mcut12\lyxdot 5to13\lyxdot 0_z0\lyxdot 15}

\caption{Those are the ratios of auto-power spectra of different stepsizes
in $256^{3}$ particle sinulations. Halos in the left plot are selected
based on mass, while halos in the right plot are chosen so that they
match with halos with 450 large steps and 5 inner steps. I need to
evaluate the effect of shot noises for each plot to see why the deviations
of 150-step simulations are larger for matched halos...}
\end{figure}
\begin{figure}
\includegraphics[width=0.5\columnwidth]{/Users/ts485/Dropbox/Tomomi/HACC/Conv/Plots4lyx/crossCoeff2_mcut12\lyxdot 5to13\lyxdot 0_z0\lyxdot 15}\includegraphics[width=0.5\columnwidth]{/Users/ts485/Dropbox/Tomomi/HACC/Conv/Plots4lyx/matchedCross1_mcut12\lyxdot 5to13\lyxdot 0_z0\lyxdot 15}

\caption{Those are the cross-correlation coefficients of different stepsizes
in $256^{3}$ particle sinulations. Halos in the left plot are selected
based on mass, while halos in the right plot are chosen so that they
match with halos with 450 large steps and 5 inner steps. I feel that
this plot somehow contradicts with Figure 11 (though this result makes
sense).}
\end{figure}

\end{itemize}

\section{Observable Box}

What we want to check/know here are:

1) What redshift can we use linear-shifting?,

2) What redshift-step size is required to preserve dynamics in simulations?
\end{document}
